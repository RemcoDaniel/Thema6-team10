\hyperlink{class_the}{The} \hyperlink{class_r_t_o_s}{R\+T\+OS} uses std\+::cout output stream for logging support.

\hyperlink{class_the}{The} \hyperlink{class_r_t_o_s}{R\+T\+OS} defines the trace macro, which can be used like cout, but prefixes each output with the current source file name and the current source line number. Hence (after the appropriate preparations) the statement


\begin{DoxyCode}
trace << \textcolor{stringliteral}{"n="} << n << \textcolor{stringliteral}{"\(\backslash\)n"};
\end{DoxyCode}


can create the output line


\begin{DoxyCode}
main.c:20 n=15
\end{DoxyCode}


This provides an easy way to check if and when a certain line of code is executed, and optionally print some debugging information.

Note that using the logging mechanism influences the execition of the task, maybe resulting in missing their deadlines. \hyperlink{class_the}{The} suggested initialization does not implement buffering, so using cout or trace can change the timing of a task that does printing considerably.

All objects (\hyperlink{class_r_t_o_s}{R\+T\+OS}, task, event, all waitables, mutex, pool, mailbox, channel) can be printed to an ostream using the $<$$<$ operator. Printing the \hyperlink{class_r_t_o_s}{R\+T\+OS} will print all \hyperlink{class_r_t_o_s}{R\+T\+OS} objects. 